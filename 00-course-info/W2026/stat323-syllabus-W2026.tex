% Options for packages loaded elsewhere
% Options for packages loaded elsewhere
\PassOptionsToPackage{unicode}{hyperref}
\PassOptionsToPackage{hyphens}{url}
\PassOptionsToPackage{dvipsnames,svgnames,x11names}{xcolor}
%
\documentclass[
  11pt,
  letterpaper,
  DIV=11,
  numbers=noendperiod]{scrartcl}
\usepackage{xcolor}
\usepackage{amsmath,amssymb}
\setcounter{secnumdepth}{-\maxdimen} % remove section numbering
\usepackage{iftex}
\ifPDFTeX
  \usepackage[T1]{fontenc}
  \usepackage[utf8]{inputenc}
  \usepackage{textcomp} % provide euro and other symbols
\else % if luatex or xetex
  \usepackage{unicode-math} % this also loads fontspec
  \defaultfontfeatures{Scale=MatchLowercase}
  \defaultfontfeatures[\rmfamily]{Ligatures=TeX,Scale=1}
\fi
\usepackage[]{mathpazo}
\ifPDFTeX\else
  % xetex/luatex font selection
\fi
% Use upquote if available, for straight quotes in verbatim environments
\IfFileExists{upquote.sty}{\usepackage{upquote}}{}
\IfFileExists{microtype.sty}{% use microtype if available
  \usepackage[]{microtype}
  \UseMicrotypeSet[protrusion]{basicmath} % disable protrusion for tt fonts
}{}
\makeatletter
\@ifundefined{KOMAClassName}{% if non-KOMA class
  \IfFileExists{parskip.sty}{%
    \usepackage{parskip}
  }{% else
    \setlength{\parindent}{0pt}
    \setlength{\parskip}{6pt plus 2pt minus 1pt}}
}{% if KOMA class
  \KOMAoptions{parskip=half}}
\makeatother
% Make \paragraph and \subparagraph free-standing
\makeatletter
\ifx\paragraph\undefined\else
  \let\oldparagraph\paragraph
  \renewcommand{\paragraph}{
    \@ifstar
      \xxxParagraphStar
      \xxxParagraphNoStar
  }
  \newcommand{\xxxParagraphStar}[1]{\oldparagraph*{#1}\mbox{}}
  \newcommand{\xxxParagraphNoStar}[1]{\oldparagraph{#1}\mbox{}}
\fi
\ifx\subparagraph\undefined\else
  \let\oldsubparagraph\subparagraph
  \renewcommand{\subparagraph}{
    \@ifstar
      \xxxSubParagraphStar
      \xxxSubParagraphNoStar
  }
  \newcommand{\xxxSubParagraphStar}[1]{\oldsubparagraph*{#1}\mbox{}}
  \newcommand{\xxxSubParagraphNoStar}[1]{\oldsubparagraph{#1}\mbox{}}
\fi
\makeatother


\usepackage{longtable,booktabs,array}
\usepackage{calc} % for calculating minipage widths
% Correct order of tables after \paragraph or \subparagraph
\usepackage{etoolbox}
\makeatletter
\patchcmd\longtable{\par}{\if@noskipsec\mbox{}\fi\par}{}{}
\makeatother
% Allow footnotes in longtable head/foot
\IfFileExists{footnotehyper.sty}{\usepackage{footnotehyper}}{\usepackage{footnote}}
\makesavenoteenv{longtable}
\usepackage{graphicx}
\makeatletter
\newsavebox\pandoc@box
\newcommand*\pandocbounded[1]{% scales image to fit in text height/width
  \sbox\pandoc@box{#1}%
  \Gscale@div\@tempa{\textheight}{\dimexpr\ht\pandoc@box+\dp\pandoc@box\relax}%
  \Gscale@div\@tempb{\linewidth}{\wd\pandoc@box}%
  \ifdim\@tempb\p@<\@tempa\p@\let\@tempa\@tempb\fi% select the smaller of both
  \ifdim\@tempa\p@<\p@\scalebox{\@tempa}{\usebox\pandoc@box}%
  \else\usebox{\pandoc@box}%
  \fi%
}
% Set default figure placement to htbp
\def\fps@figure{htbp}
\makeatother





\setlength{\emergencystretch}{3em} % prevent overfull lines

\providecommand{\tightlist}{%
  \setlength{\itemsep}{0pt}\setlength{\parskip}{0pt}}



 


\usepackage{booktabs}
\usepackage{longtable}
\usepackage{array}
\usepackage{multirow}
\usepackage{wrapfig}
\usepackage{float}
\usepackage{colortbl}
\usepackage{pdflscape}
\usepackage{tabu}
\usepackage{threeparttable}
\usepackage{threeparttablex}
\usepackage[normalem]{ulem}
\usepackage{makecell}
\usepackage{xcolor}
\KOMAoption{captions}{tableheading}
\makeatletter
\@ifpackageloaded{tcolorbox}{}{\usepackage[skins,breakable]{tcolorbox}}
\@ifpackageloaded{fontawesome5}{}{\usepackage{fontawesome5}}
\definecolor{quarto-callout-color}{HTML}{909090}
\definecolor{quarto-callout-note-color}{HTML}{0758E5}
\definecolor{quarto-callout-important-color}{HTML}{CC1914}
\definecolor{quarto-callout-warning-color}{HTML}{EB9113}
\definecolor{quarto-callout-tip-color}{HTML}{00A047}
\definecolor{quarto-callout-caution-color}{HTML}{FC5300}
\definecolor{quarto-callout-color-frame}{HTML}{acacac}
\definecolor{quarto-callout-note-color-frame}{HTML}{4582ec}
\definecolor{quarto-callout-important-color-frame}{HTML}{d9534f}
\definecolor{quarto-callout-warning-color-frame}{HTML}{f0ad4e}
\definecolor{quarto-callout-tip-color-frame}{HTML}{02b875}
\definecolor{quarto-callout-caution-color-frame}{HTML}{fd7e14}
\makeatother
\makeatletter
\@ifpackageloaded{caption}{}{\usepackage{caption}}
\AtBeginDocument{%
\ifdefined\contentsname
  \renewcommand*\contentsname{Table of contents}
\else
  \newcommand\contentsname{Table of contents}
\fi
\ifdefined\listfigurename
  \renewcommand*\listfigurename{List of Figures}
\else
  \newcommand\listfigurename{List of Figures}
\fi
\ifdefined\listtablename
  \renewcommand*\listtablename{List of Tables}
\else
  \newcommand\listtablename{List of Tables}
\fi
\ifdefined\figurename
  \renewcommand*\figurename{Figure}
\else
  \newcommand\figurename{Figure}
\fi
\ifdefined\tablename
  \renewcommand*\tablename{Table}
\else
  \newcommand\tablename{Table}
\fi
}
\@ifpackageloaded{float}{}{\usepackage{float}}
\floatstyle{ruled}
\@ifundefined{c@chapter}{\newfloat{codelisting}{h}{lop}}{\newfloat{codelisting}{h}{lop}[chapter]}
\floatname{codelisting}{Listing}
\newcommand*\listoflistings{\listof{codelisting}{List of Listings}}
\makeatother
\makeatletter
\makeatother
\makeatletter
\@ifpackageloaded{caption}{}{\usepackage{caption}}
\@ifpackageloaded{subcaption}{}{\usepackage{subcaption}}
\makeatother
\usepackage{bookmark}
\IfFileExists{xurl.sty}{\usepackage{xurl}}{} % add URL line breaks if available
\urlstyle{same}
\hypersetup{
  pdftitle={Stat 323/523: Design and Analysis of Experiments I},
  pdfauthor={Dr.~Emily Robinson},
  colorlinks=true,
  linkcolor={blue},
  filecolor={Maroon},
  citecolor={Blue},
  urlcolor={Blue},
  pdfcreator={LaTeX via pandoc}}


\title{Stat 323/523: Design and Analysis of Experiments I}
\usepackage{etoolbox}
\makeatletter
\providecommand{\subtitle}[1]{% add subtitle to \maketitle
  \apptocmd{\@title}{\par {\large #1 \par}}{}{}
}
\makeatother
\subtitle{Cal Poly - San Luis Obispo, Winter 2026}
\author{Dr.~Emily Robinson}
\date{}
\begin{document}
\maketitle


\subsection{Communication}\label{communication}

\includegraphics[width=1em,height=1em]{stat323-syllabus-W2026_files/figure-pdf/fa-icon-a7ff419e70980f9f1a65816048d94526.pdf}
Email:
\href{mailto:erobin17@calpoly.edu?subject=Stat\%20323}{erobin17@calpoly.edu}

\includegraphics[width=0.75em,height=1em]{stat323-syllabus-W2026_files/figure-pdf/fa-icon-443d06266e96702cc4dd4ffb1304f584.pdf}
Office: Building 25 Office 103 (by Statistics Department Office)

\textbf{Discord Discussion Board:}

For questions of general interest, such as course clarifications or
conceptual questions, please use the Discord page for discussion (join
via Canvas). I encourage you to give your post a concise and informative
post title/first sentence, so that other people can find it. For
example, \emph{``How do I determine the treatment structure?''} is a
better title than \emph{``help with activity 1''}.

\includegraphics[width=0.62em,height=1em]{stat323-syllabus-W2026_files/figure-pdf/fa-icon-37ad7c564657af6a696c5f96e99a441a.pdf}
While your posts are not anonymous, in this case there is no such thing
as a bad question!

\subsection{Course Logistics}\label{course-logistics}

\textbf{Class Meeting Time:} Tuesday/Thursday

\begin{itemize}
\tightlist
\item
  Section 70 at 2:10 - 4:00pm
\item
  Section 71 at 4:10 - 6:00pm
\end{itemize}

\textbf{Room:} Erhart Agriculture (Building 10-225)

\textbf{Office Hours} are held in my office (25-103) during the
following times:

\begin{longtable}[]{@{}ll@{}}
\toprule\noalign{}
Day & Time \\
\midrule\noalign{}
\endhead
\bottomrule\noalign{}
\endlastfoot
Tuesday & 9:30 - 11:00am \\
Thursday & 9:30 - 11:00am \\
\end{longtable}

Note that office hours are not just for when you have content questions.
Stop by to introduce yourself, ask questions about the broader field of
statistics, or share what you are working on!

\section{Course Description}\label{course-description}

Stat 323/523 is designed to engage you in the principles, construction
and analysis of experimental designs. Completely randomized, randomized
complete block, Latin squares, Graeco Latin squares, factorial, and
nested designs. Fixed and random effects, expected mean squares,
multiple comparisons, and analysis of covariance.

\textbf{Prerequisites}: Entrance to STAT 323/523 requires completion of
STAT 302.

\section{Learning Objectives}\label{learning-objectives}

By the end of the course, you should:

\begin{itemize}
\tightlist
\item
  understand the single factor fixed effects model, and be able to carry
  out the analysis culminating in the F-test and appropriate multiple
  comparisons.
\item
  understand the difference between fixed and random effects.
\item
  understand the rationale behind the use of blocking, Latin squares,
  and other noise-reducing designs.
\item
  be able to recognize different designs.
\item
  be able to perform the statistical computations and express the
  results of the quantitative work through your writing skills.
\end{itemize}

\section{Course Materials \& Tools}\label{course-materials-tools}

For each topic, you will prepare for class by
\includegraphics[width=1.12em,height=1em]{stat323-syllabus-W2026_files/figure-pdf/fa-icon-c79e848d9a3d708b9ec27f2a8e82e493.pdf}
watching 1-3 short lecture videos following a set of \emph{lecture slide
notes}
\includegraphics[width=0.88em,height=1em]{stat323-syllabus-W2026_files/figure-pdf/fa-icon-5ea3361430ab42477e0145b37bf037cc.pdf}
containing definitions and examples. You are expected to either print
\includegraphics[width=1em,height=1em]{stat323-syllabus-W2026_files/figure-pdf/fa-icon-dba2621d70df0db026d72b64f2337f91.pdf}
or save
\includegraphics[width=0.88em,height=1em]{stat323-syllabus-W2026_files/figure-pdf/fa-icon-20302663c48760596ae954f808544c3c.pdf}
the lecture slide notes to your device and follow along, filling
\includegraphics[width=1em,height=1em]{stat323-syllabus-W2026_files/figure-pdf/fa-icon-2eafe8fc521887db57e4cf342e07d14a.pdf}
out your set of notes while watching the videos.

During class, I will highlight any key points from the lecture videos
and we will complete activities meant to reinforce the ideas learned in
the videos.

Supplemental reading from
\includegraphics[width=0.88em,height=1em]{stat323-syllabus-W2026_files/figure-pdf/fa-icon-5ea3361430ab42477e0145b37bf037cc.pdf}
\href{http://users.stat.umn.edu/\%7Egary/book/fcdae.pdf}{A First Course
in Design and Analysis of Experiments} by Gary W. Oehlert will be
recommended.

Tools and statistical software you will need for this course:

\includegraphics[width=1.25em,height=1em]{stat323-syllabus-W2026_files/figure-pdf/fa-icon-200198f1a0ee0bd3b2404fd0ddb4d488.pdf}
Laptop

\includegraphics[width=1.25em,height=1em]{stat323-syllabus-W2026_files/figure-pdf/fa-icon-f777b47c941027ab3d8eb3c0d2458974.pdf}
Statistical software -- R/RStudio or JMP

\includegraphics[width=1em,height=1em]{stat323-syllabus-W2026_files/figure-pdf/fa-icon-fc9104599277ccb91d4e70e2088fe75b.pdf}
Gradescope -- an app on your phone or computer browser

\section{Class Schedule \& Topic
Outline}\label{class-schedule-topic-outline}

This schedule is tentative and subject to change.

\begin{figure}[H]

{\centering \includegraphics[width=1\linewidth,height=\textheight,keepaspectratio]{stat323-syllabus-W2026_files/figure-pdf/calendar-1.pdf}

}

\caption{Note: Tuesday, January 20th follows a Monday schedule.}

\end{figure}%

\begin{longtable}[]{@{}ll@{}}
\caption{Tentative schedule of class topics:}\tabularnewline
\toprule\noalign{}
Date & Topic \\
\midrule\noalign{}
\endfirsthead
\toprule\noalign{}
Date & Topic \\
\midrule\noalign{}
\endhead
\bottomrule\noalign{}
\endlastfoot
Jan 6, Jan 8 & Module 1: Intro to Design of Experiments (DOE) \\
Jan 13, Jan 15 & Module 2: Completely Randomized Designs (CRD) \\
Jan 22 & Module 3: Power \\
Jan 27, Jan 29 & Module 4: Factorials \\
Feb 5 & Module 5: Randomized Complete Block Designs (RCBD) \\
Feb 3 & Midterm Exam 1 \\
Feb 10, Feb 12 & Module 6: Random Effects \& Mixed Models \\
Feb 17, Feb 19 & Module 7: Extensions to Block Designs \\
Feb 26 & Module 8: Nonparametric Tests \\
Feb 24 & Midterm Exam 2 \\
Mar 3, Mar 5 & Final Project Worktime \\
Mar 10, Mar 12 & Final Project Presentations \\
Mar 17 & Common Final Exam \\
\end{longtable}

\newpage

\section{Course Policies}\label{course-policies}

\subsection{Assessment/Grading}\label{assessmentgrading}

Your grade in STAT 323/523 will contain the following components:

\begin{longtable}[]{@{}ll@{}}
\toprule\noalign{}
Category & Percent \\
\midrule\noalign{}
\endhead
\bottomrule\noalign{}
\endlastfoot
Class Preparatin, Participation, \& Professionalism & 5\% \\
In-class Activities & 10\% \\
Island Labs & 15\% \\
Final Project & 15\% \\
Midterm Exam 1 & 17.5\% \\
Midterm Exam 2 & 17.5\% \\
Final Exam & 20\% \\
\end{longtable}

Lower bounds for grade cutoffs are shown in the following table. I will
not ``round up'' grades at the end of the quarter. See this
\href{https://twitter.com/drseanmullen/status/1604212304622518272?s=46&t=II3oNLTSSrljVPqptoe21g}{thread}
for advice on \emph{``Playing the lines. Don't be there.''}

\begin{longtable}[]{@{}llll@{}}
\toprule\noalign{}
Letter grade & X + & X & X - \\
\midrule\noalign{}
\endhead
\bottomrule\noalign{}
\endlastfoot
A & . & 93 & 90 \\
B & 87 & 83 & 80 \\
C & 77 & 73 & 70 \\
D & 67 & 63 & 60 \\
F & \textless60 & & \\
\end{longtable}

Interpretation of this table:

\begin{itemize}
\tightlist
\item
  A grade of 85 will receive a B.
\item
  A grade of 77 will receive a C+.
\item
  A grade of 70 will receive a C-.
\item
  Anything below a 60 will receive an F.
\end{itemize}

\subsubsection{General Evaluation
Criteria}\label{general-evaluation-criteria}

In every assignment, discussion, and written component of this class,
you are expected to demonstrate that you are intellectually engaging
with the material. I will evaluate you based on this engagement, which
means that technically correct but low effort answers which do not
demonstrate engagement or understanding will receive no credit.

While this is not an English class, grammar and spelling are important,
as is your ability to communicate technical information in writing; both
of these criteria will be used in addition to assignment-specific
rubrics to evaluate your work.

\subsection{Assignment Breakdown}\label{assignment-breakdown}

\subsubsection{Class Preparation, Participation, \&
Professionalism}\label{class-preparation-participation-professionalism}

This course uses a flipped classroom model. Core content is introduced
before class through assigned videos or readings, and class time is
devoted to activities and practice. Coming to class prepared is
essential for your own learning and for your group's success.
Preparation may occasionally be checked using short, closed-note
\emph{learning quizzes} or \emph{entry/exit tickets}

Participation credit is earned by being present, prepared, and engaged
during class. You must be in class to receive participation credit. Part
of this grade also reflects professionalism, including arriving on time,
being mentally present, treating others with respect, and contributing
constructively to group work.

\begin{itemize}
\tightlist
\item
  \textbf{2} participation grades will be dropped; think of these as
  ``sick days''.
\end{itemize}

\begin{tcolorbox}[enhanced jigsaw, breakable, leftrule=.75mm, colbacktitle=quarto-callout-note-color!10!white, bottomtitle=1mm, title=\textcolor{quarto-callout-note-color}{\faInfo}\hspace{0.5em}{Note}, left=2mm, titlerule=0mm, colback=white, opacityback=0, toprule=.15mm, opacitybacktitle=0.6, toptitle=1mm, arc=.35mm, rightrule=.15mm, bottomrule=.15mm, coltitle=black, colframe=quarto-callout-note-color-frame]

If you are feeling ill, please \emph{do not come to class}. You do not
need to notify me, but instead do your best to complete the in-class
activity found on Canvas. Then check with a group member anything missed
in class. Stop by office hours after you are feeling better if you have
any questions.

\end{tcolorbox}

\subsubsection{In-class Activities}\label{in-class-activities}

Most class meetings will include in-class activities designed to
reinforce and apply course content. Class time will be dedicated to
working on these activities. You may collaborate with your group
members; however, each student must submit their own original work.

\begin{itemize}
\tightlist
\item
  In-class activities will be submitted and graded through Gradescope,
  which is integrated with Canvas.
\item
  Unless otherwise noted, activities are due at the \textbf{beginning of
  the next class meeting (2:00pm)}.
\end{itemize}

\begin{tcolorbox}[enhanced jigsaw, breakable, leftrule=.75mm, colbacktitle=quarto-callout-note-color!10!white, bottomtitle=1mm, title=\textcolor{quarto-callout-note-color}{\faInfo}\hspace{0.5em}{Note}, left=2mm, titlerule=0mm, colback=white, opacityback=0, toprule=.15mm, opacitybacktitle=0.6, toptitle=1mm, arc=.35mm, rightrule=.15mm, bottomrule=.15mm, coltitle=black, colframe=quarto-callout-note-color-frame]

Cal Poly provides access to
\href{https://www.gradescope.com/}{Gradescope}. This course will use
Gradescope for the first time this quarter, and we will work through the
process together as needed.

\end{tcolorbox}

\subsubsection{Lab Reports}\label{lab-reports}

You will complete 4 lab assignments in groups of 3-4. Each lab will
involve designing a study and collecting data, often using
\href{https://islands.smp.uq.edu.au/}{The Islands}. A written report
summarizing your methods and findings will be submitted for each lab.
More details will be provided in class.

\begin{itemize}
\tightlist
\item
  One report per group should be submitted as a PDF on Canvas.
\item
  Lab reports are tentatively due on \textbf{Thursdays at 11:59 PM}.
\end{itemize}

\subsubsection{Final Project}\label{final-project}

You will complete a paired final project in which you design and propose
an experiment, including all necessary details. The project will
culminate in a 10-minute presentation during Week 10 of the quarter.
Additional details and expectations will be provided later in the term.

\subsubsection{Exams}\label{exams}

There will be two midterm exams and a final exam, designed to assess
your individual understanding and progress in the course. You may bring
a one-sided notesheet to each midterm exams and a two-sided notesheet to
the final exam. Exam dates are listed in the course calendar. If you are
unable to attend an exam, you must \textbf{notify me in advance.}

\subsection{Late Policy}\label{late-policy}

We are living through a challenging time with unique, unusual
circumstances. I do not want class deadlines to cause you extreme stress
or anxiety. I offer \textbf{3 ``grace days''} -- 24 hours to turn in the
assignment late without a penalty. These can be used on the in-class
activities and lab assignments (a single group member must be willing to
use one of their grace days for the entire group), but not
participation, exams, or final project presentations. These ``grace
days'' can be stacked and used all at once on a single assignment or
spread out and used on separate assignments throughout the quarter.
Simply send me an email to let me know you how many ``grace days'' you
want to use on the assignment.

After using up your ``grace days'', late work will be accepted with a
20\% grade penalty for each day late (including weekends).

\begin{tcolorbox}[enhanced jigsaw, breakable, leftrule=.75mm, colbacktitle=quarto-callout-warning-color!10!white, bottomtitle=1mm, title=\textcolor{quarto-callout-warning-color}{\faExclamationTriangle}\hspace{0.5em}{Automatic Canvas Settings}, left=2mm, titlerule=0mm, colback=white, opacityback=0, toprule=.15mm, opacitybacktitle=0.6, toptitle=1mm, arc=.35mm, rightrule=.15mm, bottomrule=.15mm, coltitle=black, colframe=quarto-callout-warning-color-frame]

Canvas is set up to automatically input 0\% for missing assignments (as
an incentive to go complete the assignment) and apply the 20\% grade
deduction policy. I will need to manually adjust your grade when you use
your grace days so it is important for you to leave a note on your
assignment and email me. You are responsible for double checking your
grade.

\end{tcolorbox}

If you find yourself with extenuating circumstances beyond the defined
late policy, please email me before the due date.

\section{Course Expectations}\label{course-expectations}

You will get out of this course what you put in. The following excerpt
was taken from Rob Jenkins' article ``Defining the Relationship'' which
was published in The Chronicle of Higher Education (August 8, 2016).
This accurately summarizes what I expect of you in my classroom (and
also what you should expect of me).

\emph{``I'd like to be your partner. More than anything, I'd like for us
to form a mutually beneficial alliance in this endeavor we call
education.}

\emph{I pledge to do my part. I will:}

\begin{itemize}
\tightlist
\item
  \emph{Stay abreast of the latest ideas in my field.}
\item
  \emph{Teach you what I believe you need to know; with all the
  enthusiasm I possess.}
\item
  \emph{Invite your comments and questions and respond constructively.}
\item
  \emph{Make myself available to you outside of class (within reason).}
\item
  \emph{Evaluate your work carefully and return it promptly with
  feedback.}
\item
  \emph{Be as fair, respectful, and understanding as I can humanly be.}
\item
  \emph{If you need help beyond the scope of this course, I will do my
  best to provide it or see that you get it.}
\end{itemize}

\emph{In return, I expect you to:}

\begin{itemize}
\tightlist
\item
  \emph{Show up for class each day or let me know (preferably in
  advance) if you have some good reason to be absent.}
\item
  \emph{Do your reading and other assignments outside of class and be
  prepared for each class meeting.}
\item
  \emph{Focus during class on the work we're doing and not on extraneous
  matters (like whoever or whatever is on your phone at the moment).}
\item
  \emph{Participate in class discussions.}
\item
  \emph{Be respectful of your fellow students and their points of view.}
\item
  \emph{In short, I expect you to devote as much effort to learning as I
  devote to teaching.}
\end{itemize}

\emph{What you get out of this relationship is that you'll be better
equipped to succeed in this and other college courses, work-related
assignments, and life in general. What I get is a great deal of
professional and personal satisfaction. Because I do really like you
{[}all{]} and want the best for you.''}

\section{Learning Environment and
Support}\label{learning-environment-and-support}

I am committed to creating a safe and inclusive learning environment
where all students feel respected and supported. If there are any ways I
can improve the classroom environment to make it more welcoming for you,
please don't hesitate to let me know.

If you have a disability and require accommodations to fully participate
in the course, please contact me as soon as possible to discuss how I
can best support you. I also encourage you to register with Cal Poly's
Disability Resource Center (Building 124, Room 119 or at 805-756-1395)
to explore additional accommodations that may be available to you.

If you are experiencing food insecurity, housing instability, or other
challenges that may impact your ability to succeed in this course,
please refer to the resources listed on Canvas under ``Student Support
Services at Cal Poly.'' These resources provide a range of essential
support services, including emergency financial assistance, counseling,
and academic support.

I am committed to working with you to ensure that you have the resources
and support you need to succeed in this course. Let's work together to
create a positive and inclusive learning environment for all students.

\section{Academic Integrity and Class
Conduct}\label{academic-integrity-and-class-conduct}

Academic integrity is a fundamental value of this course and of the
university. Simply put, I will not tolerate cheating, plagiarism, or any
other form of academic dishonesty.

Any incident of academic misconduct, including dishonesty, copying, or
plagiarism, will be reported to the Office of Student Rights and
Responsibilities.

Cheating will earn you a grade of 0 on the assignment and an overall
grade penalty of at least 10\%. In circumstances of flagrant cheating,
you may be given a grade of F in the course.

It is important to note that paraphrasing or quoting another's work
without proper citation is a form of academic misconduct. This includes
using Chat GPT, which should only be used to generate ideas, provide
feedback suggestions to improve your lab reports, and not as a
substitute for your own work and writing. See more details below.

To ensure academic integrity, please be sure to cite all sources
appropriately and only use Chat GPT in an ethical manner. For more
information on academic misconduct and what constitutes cheating and
plagiarism, please see
\href{https://academicprograms.calpoly.edu/content/academicpolicies/Cheating}{academicprograms.calpoly.edu/content/academicpolicies/Cheating}.

\section{ChatGPT}\label{chatgpt}

Use AI as a glorified tutor, not a ghostwriter. In this course, the
value of AI tools is in helping you learn, practice, and clarify. They
are not meant to produce work for you. Interpretation and critical
thinking are important and simply copying AI output will not meet
expectations and may constitute academic misconduct.

\textbf{What's encouraged:}

\begin{itemize}
\tightlist
\item
  Concept help \& explanations: Ask AI to re-explain ideas in plain
  language, give alternative examples, or break down statistical terms.
\item
  R support: Use AI to clarify error messages, suggest functions, or
  outline strategies. Always run, test, and annotate the code yourself.
\item
  Feedback on your work: Ask AI to critique your draft code or writing
  for clarity, not to write an introduction for you.
\end{itemize}

\textbf{What's not allowed:}

\begin{itemize}
\tightlist
\item
  No AI-written answers submitted as your own. Interpretation is your
  responsibility.
\item
  No copying/pasting code or text without understanding. Code can fail
  silently or even introduce harmful commands. You should always
  understand and adapt before running anything.
\item
  No quiz use. In-person quizzes are closed note/book, and AI is
  prohibited.
\item
  No uncredited use. Any substantive AI assistance must be disclosed
  (see below).
\end{itemize}

\textbf{CSU System Policy}

Cal Poly and all CSU campuses have access to ChatGPT Edu, provided
through the CSU system. This version:

\begin{itemize}
\tightlist
\item
  Provides advanced AI capabilities tailored for CSU students and
  employees.
\item
  Includes privacy, security, and data protections (similar to email,
  Google Workspace, and Microsoft 365).
\item
  Offers single sign-on (SSO) and campus workspaces for sharing.
\item
  Protects confidentiality, though CSU has legal/operational obligations
  that may require access to user data. These activities are not
  intended to monitor legitimate academic use.
\end{itemize}

For details, see:

\begin{itemize}
\tightlist
\item
  \href{https://genai.calstate.edu/guidelines-safe-and-responsible-use-generative-ai-tools}{CSU
  Safe and Responsible Use Guidelines}
\item
  \href{https://genai.calstate.edu/communities/students}{CSU Student
  Community Resources}
\end{itemize}

\textbf{Your responsibilities:}

\begin{itemize}
\tightlist
\item
  Think critically. If you can't explain a result in your own words, you
  don't understand it.
\item
  Demonstrate engagement. Work that is technically correct but shows no
  reasoning or interpretation will not receive full credit.
\item
  Acknowledge \emph{extensive} AI use. At the end of any assignment
  where you used AI, include a brief disclosure. Example:
\end{itemize}

\begin{quote}
AI assistance: Consulted ChatGPT for debugging the \texttt{lm()}
function syntax and clarifying the difference between a CRD and RCBD.
\end{quote}

\textbf{Reminder:} In-person exams and verbal presentations exist for a
reason. This class emphasizes your connection of research questions to
experimental designs and statistical interpretations.




\end{document}
